%% start of file `template.tex'.
%% Copyright 2006-2010 Xavier Danaux (xdanaux@gmail.com).
%
% This work may be distributed and/or modified under the
% conditions of the LaTeX Project Public License version 1.3c,
% available at http://www.latex-project.org/lppl/.

% Version: 20110122-4

\documentclass[11pt,letter,nolmodern]{moderncv}

\usepackage{cv}

\usepackage[english]{babel}
\linespread{0.9}

% personal data
\title{Robotics Engineer}
\extrainfo{%
\octocat~\httplink{github.com/tdenewiler}\\%
} % optional, remove the line if not wanted

\myquote{We are what we repeatedly do. Excellence, then, is not an act, but a habit.}{Aristotle}

%\nopagenumbers{} % uncomment to suppress automatic page numbering for CVs longer than one page

\begin{document}
%\setmainfont{Minion Pro}
%\setsansfont{Myriad Pro}

\hyphenpenalty=10000
\maketitle

\section{Skills}

\subsection{Expert Skills}
\cvline{Systems Engineering}{Systems Integration, Development Models, Field Experiments}
\cvline{Hardware}{Robotic Sensors (Perception, Localization)}
\cvline{Algorithm Development}{Autonomous Vehicle Testing, Autonomy Architectures, Sensor Fusion,
        Kalman Filters, Vehicle Control}

\subsection{Development}
\cvcomputer{Languages}{C, C++, Python, Shell/Bash}
           {Tools}{Docker, Atlassian, GitHub, GitLab, Jenkins, static analysis}
% \cvcomputer{Operating Systems}{GNU/Linux (Debian, Ubuntu, Mint), MacOS X, Windows}
%            {Robotics Frameworks}{Robot Operating System (ROS), Unmanned Maritime Autonomy Architecture (UMAA)}

% \subsection{Office and Tools}
% \cvcomputer{Documents}{Markdown, \TeX, \LaTeX}
%            {Office}{Inkscape, draw.io, Gimp}

\section{Experience}
\subsection{NIWC Pacific}

\tlcventry{2024}{2021}{Engineering Manager, Autonomy Performance Analysis}{}{}{}{%
\begin{itemize}
  \item Lead team of 9 to develop analysis architecture to provide timely
        and actionable results to engineers and management.
  \item Architecture based on hierarchy of capabilities organized by protocol.
        Allows for inheritance of results and intuitive reporting.
  \item Developed metrics to accurately describe capability performance.
        Facilitates comparison of results across system evolution.
  \item Integrate work with multiple organizations doing vehicle autonomy and analysis.
        Distributed our products as Docker images and Python packages with well-defined and
        documented interfaces.
  \item Documentation and software artifacts automatically generated during
        continuous integration.
\end{itemize}}

\tlcventry{2024}{2021}{Senior Engineer, Mission Management}{}{}{}{%
\begin{itemize}
  \item AI planning for long duration missions.
  \item Develop translation modules between software frameworks (ROS 2 and UMAA) to enable use
        of open source software.
        Unit and integration tests written for translation and algorithm execution.
\end{itemize}}

\tlcventry{2024}{2021}{Senior Engineer, Division Technical Development Lead}{}{}{}{%
\begin{itemize}
  \item Software Technical Development Lead for Autonomous Technologies Division
        (>500 employees, >100 software developers).
  \item Created Software Community of Interest.
        In coordination with Division project leads, developed example projects to highlight
        and share best practices for software development.
        Met with projects throughout the Division to help them tailor best practices for their workflow.
\end{itemize}}

\tlcventry{2020}{2012}{Engineering Manager, Autonomous Vehicle Development}{}{}{}{%
\begin{itemize}
  \item Vehicles: Assault Amphibious Vehicle, HMMWV, Ford Escape Hybrid, QuadSki
  \item Domains: Off-road, unimproved roads, beaches, surf zone, open ocean, suburban
  \item Low-cost autonomy for large vehicles in congested and rugged terrain with passive capabilities.
  \item Setup and managed continuous integration environment for program with >50 developers from >10 distributed
        organizations and >100 git repositories.
        Used standard and newly developed test methods to characterize large autonomous vehicle performance.
        Integrated high- and low-fidelity simulation with validated results for configured scenarios.
        Developed architecture to support running on vehicle, in simulation, and with sensor playback.
  \item Responsible for reviewing and merging proposed changes into baseline software configuration.
        Documented and communicated clear pathway to integrate into and evolve baseline configuration.
        Developed test methods and metrics to check module and system performance.
  \item Set up automated builds (static analysis, unit tests, regression tests).
        Focused on rapid feedback for faster iterations:
        build and test subsets of modules, configure dependencies for speed, faster-than-real-time simulation.
  \item Automated builds started at 4-8 hours to complete.
        With architecture re-factor and utilities to identify changed modules and dependent modules the duration
        was reduced to 15 minutes to 3 hours.
        These builds ran on average 3 times per day.
  \item Developed RPM-based PID throttle controller with switching on pitch angle used in surf zone.
        Contributor to AAV motion model development (trinary steering, dual thrust system) and validation.
  \item Outfitted Ford Escape Hybrid with large number of perception and localization sensors and computing.
  \item Directed implementation of unsupervised learning algorithm for lidar calibration and mesh-based object
        segmentation of lidar data.
\end{itemize}}

\tlcventry{2011}{2009}{Systems Engineer, EOD Robotics}{}{}{}{%
\begin{itemize}
  \item Improved Kalman filter for localization using coordinate ascent machine learning.
  \item Implemented control Lyapunov function-based algorithm for waypoint navigation and showed significant
        improvement over PID controller.
  \item Added use of CMake macros and functions to automony software build system to simplify
        addition of new modules to architecture.
\end{itemize}}

\tlcventry{2009}{2007}{Lead Engineer, UAV-UGV Refueling}{}{}{}{%
\begin{itemize}
  \item Developed UAV landing pad for large UGV that centers and refuels UAV to continue operations.
  \item Wrote interface to UAV to control flight via waypoints from OCU\@.
\end{itemize}}

\tlcventry{2008}{2001}{Mechanical Engineer, Large Patrol UGV}{}{}{}{%
\begin{itemize}
  \item Managed wireless communications infrastructure for mobile robots.
  \item Created tools to map wireless signal strength and GPS satellite observability.
  \item Rapid prototyping of novel large UGV hardware (marsupial capability, UAV landing/refueling pad, automatic
        gate operation).
  \item Supported large number of system test events at locations throughout U.S.
\end{itemize}}

%\pagebreak

\subsection{Other Experience}

\tlcventry{2024}{2010}{Developer/Maintainer}{Open Source Projects}{}{}{%
\begin{itemize}
  \item \textit{Statick}: static analysis and linting framework easily customizable for warnings and reports.
        Automated pipelines used for pull requests and releases to PyPI.
  \item \textit{ROS Example Node}: minimal working example of common ROS concepts and best practices.
\end{itemize}}

\tlcventry{2015}{2011}{Engineering Mentor}{University of California, San Diego}{}{}{%
  %Computer Science Department, Prof.\ Ryan Kastner's Lab
\begin{itemize}%
  \item Mentored student teams with undergraduate and graduate students in areas of visual odometry, SLAM, stereo
        vision, Kinect for skeleton and gesture tracking, Turtlebot navigation, and underwater vehicles.
  \item Provided guest lectures on large ground vehicle autonomy development emphasizing test and evaluation.
\end{itemize}}

\tlcventry{2011}{2008}{Software Lead, Stingray UUV}{University of California, San Diego}{}{}{%
  %Stingray Underwater Vehicle
\begin{itemize}%
  \item Participated in AUVSI UUV student competition.
  \item Developed: custom networking software, computer vision algorithms to perceive underwater obstacle course,
        PID algorithms for UUV with six degrees of freedom motion.
\end{itemize}}

% \tlcventry{2007}{2006}{Consultant}{South Park Systems}{}{}{%
% \begin{itemize}
%   \item Provided Linux administration and C programming support for project working with Neurosciences Institute.
%   \item Main work was getting robot simulator working with hardware-in-the-loop testing.
% \end{itemize}}

%\pagebreak

\section{Education}

\tlcventry{2011}{2009}{Master's, Mechanical Engineering}{University of California, San Diego}{}{}{%
\begin{itemize}
  \item Focus on Controls \& Estimation
  \item Thesis: \textit{Improving Autonomous Navigation in EOD Robots}
  \item Course notes available on GitHub
\end{itemize}}

% \tldatecventry{2004}{C/C++ Programming}{University of California Extension, San Diego}{}{}{Introductory Course}

\tlcventry{2000}{1996}{Bachelor's, Mechanical Engineering}{University of California, San Diego}{}{}{}

% \section{Personal interests}
%
% \cvhobby{Sports}{Volleyball, Basketball, Hiking, Swimming, Paddle Boarding}
% \cvhobby{Volunteer}{Mentor in Big Brothers, Big Sisters from 2005 --- 2009}
% \cvhobby{Others}{Traveling, Reading, Gardening}

\end{document}
