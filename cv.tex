%% start of file `template.tex'.
%% Copyright 2006-2010 Xavier Danaux (xdanaux@gmail.com).
%
% This work may be distributed and/or modified under the
% conditions of the LaTeX Project Public License version 1.3c,
% available at http://www.latex-project.org/lppl/.

% Version: 20110122-4

\documentclass[11pt,letter,nolmodern]{moderncv}

\usepackage{cv}

\usepackage[english]{babel}
\linespread{0.9}

% personal data
\title{Robotics Engineer}
\extrainfo{%
\octocat~\httplink{github.com/tdenewiler}\\%
} % optional, remove the line if not wanted

\myquote{We are what we repeatedly do. Excellence, then, is not an act, but a habit.}{Aristotle}

%\nopagenumbers{} % uncomment to suppress automatic page numbering for CVs longer than one page

\begin{document}
%\setmainfont{Minion Pro}
%\setsansfont{Myriad Pro}

\hyphenpenalty=10000
\maketitle

\section{Skills}

\subsection{Expert Skills}
\cvline{Hardware}{Robotic Sensors (Perception, Localization)}
\cvline{Algorithm Development}{Autonomous Systems Testing, Autonomy Architectures,  Sensor Fusion,
        Kalman Filters, Vehicle Control}
\cvline{Systems Engineering}{Development Models, Field Experiments}

\subsection{Development}
\cvcomputer{Languages}{C, C++, Python, Shell/Bash}
           {Tools}{Docker, CMake, Doxygen, Atlassian, GitHub, GitLab, Jenkins, static analysis}
\cvcomputer{Operating Systems}{GNU/Linux (Debian, Ubuntu, Mint), MacOS X, Windows}
           {Robotics Frameworks}{Robot Operating System, Unmanned Maritime Autonomy Architecture}

\subsection{Office and Tools}
\cvcomputer{Documents}{Markdown, \TeX, \LaTeX}
           {Office}{Inkscape, draw.io, Gimp}

\section{Experience}
\subsection{Robotics Experience}

\tlcventry{2021}{0}{Senior Engineer}{NIWC Pacific}{San Diego}{}{%
  Large Unmanned Surface Vehicle
\begin{itemize}
  \item 150-300 foot vessels.
  \item Smaller surrogate vessels.
  \item Mission management development for long-duration missions with AI planning.
  \item Create test methods to characterize system performance.
        Focus on navigation capabilities.
\end{itemize}}

\tlcventry{2010}{0}{Developer/Maintainer}{}{}{}{%
  Open Source Projects
\begin{itemize}
  \item \textit{Statick}: static analysis and linting framework easily customizable for warnings and reports.
        Automated pipelines used for pull requests and releases to PyPI.
  \item \textit{ROS Example Node}: minimal working example of common ROS concepts and best practices.
\end{itemize}}

\tlcventry{2024}{2021}{Senior Engineer}{NIWC Pacific}{San Diego}{}{%
  Software Technical Development Lead for Autonomous Technologies Division
\begin{itemize}
  \item Created Community of Interest for Division (>500 employees, >100 software developers).
  \item Created example projects to highlight and share best practices for software development.
  \item Met with projects throughout the Division to help them tailor best practices for their workflow.
\end{itemize}}

\tlcventry{2020}{2018}{Senior Engineer}{NIWC Pacific}{San Diego}{}{%
  Autonomous Assault Amphibious Vehicle
\begin{itemize}
  \item Develop autonomy on 29 ton amphibious vehicle operating on shore and through surf zone.
  \item Small team with everyone working on all systems.
  \item RPM-based PID throttle controller with switching on pitch angle used in surf zone.
\end{itemize}}

\tlcventry{2017}{2013}{Principal Investigator}{SPAWAR Pacific}{San Diego}{}{%
  ONR Code 30 Ground Autonomy Systems Integration
\begin{itemize}
  \item Low-cost autonomy for large off-road ground vehicles in congested and rugged terrain  with passive capabilities.
  \item Responsible for reviewing and merging proposed changes into baseline software configuration.
  \item Major contributor to develop Autonomy Architecture Reference Model.
  \item Using standard and newly developed test methods to characterize large autonomous ground vehicle performance.
  \item Setup and manage continuous integration environment for program with >50 developers from >10 distributed
        organizations and >100 git repositories.
  \item Set up automated builds (static analysis, unit tests, regression tests) with results feedback using Jenkins
        and Atlassian Stash.
\end{itemize}}

\tlcventry{2012}{2011}{Lead Engineer}{SPAWAR Pacific}{San Diego}{}{%
  Long Range Obstacle Detection
\begin{itemize}
  \item Outfitted Ford Escape Hybrid with large number of perception and localization sensors and computing.
  \begin{itemize}
    \item Perception: Velodyne lidar, Ibeo lidar, Delphi automotive radar, GigE stereo cameras, FLIR stereo cameras.
    \item Localization: Novatel DGPS, DGPS serial radio, Microstrain IMU, GINA IMU, KVH gyro.
    \item Computing: Installed rack, rackmount servers, and power distribution, created read-only Linux filesystem.
  \end{itemize}
  \item Ported autonomy algorithms from ACS to ROS\@.
  \item Created URDF and visualization tools for system from SolidWorks 3D CAD models.
  \item Implemented joystick teleoperation of vehicle.
  \item Directed implementation of unsupervised learning algorithm for lidar calibration and mesh-based object
        segmentation of lidar data.
\end{itemize}}

\tlcventry{2011}{2009}{Systems Engineer}{SPAWAR Pacific}{San Diego}{}{%
  EOD Robotics Autonomy Developer
\begin{itemize}
  \item Improved Kalman filter for localization using coordinate ascent machine learning.
  \item Implemented control Lyapunov function-based control algorithm for waypoint navigation and showed significant
        improvement over PID controller.
  \item Added use of CMake macros and functions to Autonomous Capabilities Suite build system to greatly simplify
        addition of new modules to architecture.
  \item Installed Trac on main development server for management and developer use.
\end{itemize}}

\tlcventry{2009}{2007}{Lead Engineer}{SAIC at SPAWAR Pacific}{}{}{%
  Autonomous UAV-UGV Refueling
\begin{itemize}
  \item Developed UAV landing pad for large UGV that centers and refuels UAV and allows UAV to continue operations.
  \item Wrote JAUS interface to UAV to control flight via waypoints from OCU\@.
\end{itemize}}

\tlcventry{2008}{2001}{Mechanical Engineer}{SAIC at SPAWAR Pacific}{}{}{%
  Mobile Detection, Assessment and Response System (Ground)
\begin{itemize}
  \item Managed wireless communications infrastructure for mobile robots.
  \item Created tools to map wireless signal strength and GPS satellite observability.
  \item Rapid prototyping of novel large UGV hardware (marsupial capability, UAV landing/refueling pad, automatic
        gate operation).
  \item Supported large number of system test events at remote locations throughout U.S.
\end{itemize}}

\subsection{Other Experience}

\tlcventry{2015}{2011}{Engineering Mentor}{University of California}{San Diego}{}{%
  Computer Science Department, Prof.\ Ryan Kastner's Lab
\begin{itemize}%
 \item Mentored student teams with undergraduate and graduate students in areas of visual odometry, SLAM, stereo
       vision, Kinect for skeleton and gesture tracking, Turtlebot navigation, and underwater vehicles.
 \item Provided guest lectures on large ground vehicle autonomy development emphasizing test and evaluation.
\end{itemize}}

\tlcventry{2011}{2008}{Software Lead}{University of California}{San Diego}{}{%
  Stingray Underwater Vehicle
\begin{itemize}%
 \item Wrote custom networking software for autonomous underwater vehicle to compete in AUVSI UUV student competition.
 \item Wrote computer vision algorithms to perceive underwater obstacle course environment.
 \item Wrote and tuned PID algorithms for UUV control system with six degrees of freedom motion.
 \item Ported initial algorithms to ROS\@.
\end{itemize}}

\tlcventry{2007}{2006}{Consultant}{South Park Systems}{}{}{%
  Robotics
\begin{itemize}
  \item Provided Linux administration and C programming support for project working with Neurosciences Institute.
  \item Main work was getting robot simulator working with hardware-in-the-loop testing.
\end{itemize}}

\section{Education}

\tlcventry{2011}{2009}{Master's, Mechanical Engineering}{University of California, San Diego}{}{}{%
\begin{itemize}
  \item Focus on Controls \& Estimation
  \item Thesis: \textit{Improving Autonomous Navigation in EOD Robots}
  \item Course notes available on GitHub
\end{itemize}}

% \tldatecventry{2004}{C/C++ Programming}{University of California Extension, San Diego}{}{}{Introductory Course}

\tlcventry{2000}{1996}{Bachelor's, Mechanical Engineering}{University of California, San Diego}{}{}{}

\section{Personal interests}

\cvhobby{Sports}{Volleyball, Basketball, Hiking, Swimming, Paddle Boarding}
\cvhobby{Contributions}{Statick, Robot Operating System, UCSD Lecture Notes (on GitHub)}
\cvhobby{Volunteer}{Mentor in Big Brothers, Big Sisters from 2005 --- 2009}
\cvhobby{Others}{Traveling, Reading, Gardening}

\end{document}
